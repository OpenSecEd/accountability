\question[3]\label{q:accountability}
% tags: accountability:E:C
Explain the idea of double-entry book-keeping.

\begin{solution}
  It originates from banks.
  Every entry is either a credit or a debit.
  Every credit must have a corresponding debit, i.e.\ they cancel each other if 
  added together.
  This means that when all entries are added together, the final balance should 
  be zero.
  Thus, we keep the constant state of zero balance, and when the final balance 
  is non-zero we know that something is wrong.
\end{solution}


\question\label{q:accountability}
% tags: accountability:E
\begin{parts}
  \part[2] Explain what dual control is.
  \begin{solution}
    In dual control you must be two or more staff members acting on 
    a transaction together to authorize it.
    I.e.\ at the same time.
  \end{solution}

  \part[3] How does dual control differ from functional separation?
  \begin{solution}
    Functional separation means that two or more staff members must act on 
    a transaction at different points in the transaction path.
    For example one adds the transaction and another approves it.
    Compare this to the two having to act together at the same time.
  \end{solution}
\end{parts}


\question[3]\label{q:accountability}
% tags: accountability:E:C
What is the purpose of separation of duty?

\begin{solution}
  The purpose of separation of duty is to make it more difficult for 
  a malicious entity to subvert the system.
  E.g.\ a systems developer at the bank should not be able to add a back-door 
  for himself so that he later can steal money without anyone noticing.
  If he is only allowed to write the code, but not to certify its correct 
  function --- then there is a higher chance that he is caught before the 
  system is launched.
  So to subvert the system a malicious actor must act together with others.
\end{solution}


\question[6]\label{q:accountability}
% tags: accountability:C:A
The Clark-Wilson security policy model was designed to ensure integrity in 
a system, briefly explain its key points.

\begin{solution}
  The Clark-Wilson model outlines how to design a system which guarantees 
  integrity.
  This means that a system that fulfils the Clark-Wilson requirements will be 
  secure against malicious attempts to put it into an illegal state.

  It first differentiates between two types of data: untrusted data and trusted 
  data.
  Then it requires a finite number of transformations to operate on the data, 
  never allow any entity to operate on the data directly.
  The transformations must be able to transfer untrusted data to trusted data 
  --- or discard it.
  Then the subjects are allowed to use a finite subset of the transformations.
  Furthermore it requires separation of duties for setting up the 
  access-control lists for and executing the transformations.
\end{solution}


\question[2]\label{q:accountability}
What is the purpose of logging?

\begin{solution}
  The purpose of logging is to be able to follow how the system has 
  transitioned between states.
  We want to do this to be able to find vulnerabilities that might have been 
  exploited during a breach.
  Also to verify or reject possible breaches.
\end{solution}


\question[3]\label{q:accountability}
% tags: accountability:E:C
Give a brief overview of the most important aspects of a secure logging system.

\begin{solution}
  The logging system must be stored in a secure location.
  It must be append only --- i.e.\ no one can read or change the log.
  There are problems with the administrator of a system, since he or she can 
  modify the logging system and thus \enquote{hide the tracks}.
  This must be solved by setting up the logging system with separation of 
  duties, e.g.\ the logging system of system A is under control of 
  administrator B and the logging system of system B is under control of 
  administrator A.
\end{solution}


\question[5]\label{q:accountability}
% tags: accountability:C:A
Sometimes it is impossible to have a system online all the times.
Outline an idea for how you can design a secure logging system for such 
a system.

\begin{solution}
  We can use cryptography.
  We can encrypt and sign each log entry, then we can link the log entries to 
  each other so we can verify the order.
  This way we can detect changes in the log.
  The can use a one-way function to construct a chain of keys where we can only 
  go forward, so we cannot find previous keys.
\end{solution}
