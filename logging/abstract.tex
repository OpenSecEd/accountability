\emph{Summary:}
Another part of accountability is logging.
This is important to detect attacks by analysing how events transpired.

\emph{Intended learning outcomes:}
In particular, the \acp{ILO} are that you are able to:
\begin{itemize}
  \item \emph{evaluate} advantages and disadvantages of different approaches to 
    audit trails,
  \item \emph{analyse} a situation and \emph{design} proper logging.
\end{itemize}

\emph{Reading:}
Anderson describes logging (or audit trails) in Chapter 
10 \enquote{Banking and Bookkeeping} in 
\citetitle{Anderson2008sea}~\cite{Anderson2008sea}.
We will also discuss the secure logging system of 
\citeauthor{schneier1999secure}~\cite{schneier1999secure} as an example of how 
to achieve secure logging in a challenging environment.
The construction described therein is a method to safely store audit logs in an 
untrusted machine; in the scheme, all log entries generated prior to 
a compromise will be impossible for the attacker to read, modify, or destroy 
undetectably.
